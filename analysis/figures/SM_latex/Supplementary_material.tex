\documentclass{article}
\usepackage{graphicx}
\usepackage[format=plain,labelfont=it]{caption}
\usepackage[left=2.5cm, right=2.5cm, top=2.5cm,bottom=2.5cm]{geometry}

\title{Supplementary Material\\
	\vspace{1em}
	Increased Generalization in a Peak Procedure after Delayed Reinforcement}
\author{Jonathan Buriticá\\ \& \\ Emmanuel Alcalá}
\date{}

\begin{document}

\maketitle
The Equation 3 of the main text can be written as:

\[
	A = \sum_{i=1}^{3}d_i\times \mid r - r_i \mid
\]

The Figure \ref{fig:fig1} shows a visual representation of this area to be maximized by an exhaustive algorithm. Because the product of durations $d_i$ and the absolute difference $\mid r - r_i \mid$ are technically areas, we must find the $start$ and $stop$ (or the lower and upper limits of $d_2$) such that the sum of the three areas is maximized.

\begin{figure}[ht]
	\centering
	\includegraphics[scale=0.8]{SM_figure_1.pdf}
	\caption{Visual representation of the Equation 3. $r_1, r_2$ and $r_3$ are the response rate states; $d_1, d_2$ and $d_3$ their durations.}
	\label{fig:fig1}
\end{figure}

\begin{figure}[ht]
	\centering
	\includegraphics[width=0.8\textwidth]{SM_figure_2.pdf}
	\caption{Lever response acquisition curves per subject and component. Every symbol represents the same subject. The data shows a similar acquisition rate on both components. Sessions progressed until the rats reached the acquisition criteria (see the section Procedure for details).}
\end{figure}

\begin{figure}[ht]
	\centering
	\includegraphics[width=\textwidth]{SM_figure_3.pdf}
	\caption{Distribution of time intervals between every response and the next reinforcer by for the baseline phase. The components are shown in different colors.}
\end{figure}
	
\begin{figure}[ht]
	\centering
	\includegraphics[width=\textwidth]{SM_figure_4.pdf}
	\caption{Distribution of time intervals between every response and the next reinforcer for the experimental phase. The intervals of the delayed component (grey) have a bimodal distribution with maximum values at six and aproximately 20 seconds in all subjects except M332.}
\end{figure}

\begin{figure}[ht]
	\centering
	\includegraphics[width=0.8\textwidth]{SM_figure_5.pdf}
	\caption{Example trial for the calculation of the FWHM metric and the peak for subject M331 in the experimental phase, last session. The arrow in the delayed component (left panel) shows the ramp part of the generalization gradient, which can be safely ignored in the calculation of the FWHM. The rug (vertical lines) in the bottom of the plots shows the individual responses that served for the computation of the density function.}
\end{figure}

\end{document}